\section{UART から HDMIへ}
Tang Nano 9K の HDMI 出力端子を有効に使うために UART からの
入力を HDMI へと出力するシステムを考えます。
HDMI への出力は Tang Nano 9K の example \footnote{github のレポジトリ} にある 
picotiny を利用します。picotiny は RISC-V を使用した HDMI 出力を含む
典型例として example に収録されており、その中の svo\_hdmi\_top.v を
ほぼそのまま利用することで簡単に VGA の表示ができます。
名称の
SVO
\footnote {
	\url{https://github.com/cliffordwolf/SimpleVOut}
}
は Simple Video Output の略のようで
Claire Wolf 氏が Xilinx 用に開発したオープンソースの
モジュール群です。そのため、インタフェースが AXI Stream になっています。

\subsection{Polyphony でコンバート}
Polyphony の役目はインタフェースの変換です(図\ref{polyphony-conv})。

UART で入力し HDMI へと出力する全体の流れを図\ref{uart-to-hdmi}に
示します。UART の RX で受けたアスキーコードは、
一旦 UART MASTER に蓄えられます。Polyphony で書いたモジュールは
簡易バス・インターフェースを用いてコードを読み込みます。
その後、SVO に対し AXI Stream を通してコードを送ります(図\ref{polyphony-conv})。
出力は HDMI 用の4組の信号線で、TMDS というプロトコルを使用し、
最終的には基板外へのHDMI出力信号となります。

UART の入力は Python で書くことが出来ました。
AXI Stream へのデータの引き渡しも、簡易的には Python で書くことが出来ます。
ただし、AXI Stream は細かい規定があり、
そのすべてを満足しようとするのは現在の Polyphony では難しく、
あくまでこのシステムでのみ動くコードになっていることに注意が必要です。
より細かいタイミングを Python で記述しようとする場合は
開発中の Polyphony (github の devel のブランチ) を使う必要があります。

\figureimage{uart-to-hdmi}{UART から HDMI への全体図}
\figureimage{polyphony-conv}{Polyphony で変換するインタフェース図}

\subsection{AXI Stream でアスキーコード入力を受け付ける}
SVO は AXI Stream の入力としてアスキーコードを受け付けて画面に表示する
機能を持っています。
AXI Stream は 8bit の term\_in\_tdata にアスキーコードを設定した上で、
term\_in\_tvalid にデータ有効であることを示すために 1 を書き込みます。
term\_in\_tvalid が 1 になったタイミングで、SVO の term\_out\_tready が
1 であれば、データが受け付けられたと解釈します。


\subsection{AXI Stream について}
TBD

\subsection{HDMI と UART のインタフェースをトップモジュールとしてつくる}
まずは段階的に実現します。
あらたに Gowin の IDE でプロジェクトを作りトップモジュールを作ります(リスト\ref{uart-hdmi-top})。
ここではプロジェクト名を uart-hdmi としました。合わせて物理制約ファイルも
掲げます(リスト\ref{uart-hdmi-cst})。

\begin{table}
\label{uart-hdmi-top}
\caption{UART と HDMI の I/F を含むトップモジュール}
\begin{C++}
module top(
    input wire clk,

    output wire uart_tx,
    input wire uart_rx,

    output wire tmds_clk_n,
    output wire tmds_clk_p,
    output wire [2:0] tmds_d_n,
    output wire [2:0] tmds_d_p,

    input wire sw1_i,
    input wire sw2_i,
    output wire [5:0] led_o
);
\end{C++}
\end{table}

\begin{table}
\label{uart-hdmi-cst}
\caption{UART と HDMI のI/F 用の物理制約ファイル}
\begin{C++}
IO_LOC "clk" 52;
IO_PORT "clk" IO_TYPE=LVCMOS33 PULL_MODE=UP;

// UART
IO_LOC "uart_tx" 17;
IO_PORT "uart_tx" IO_TYPE=LVCMOS33 PULL_MODE=UP DRIVE=8;
IO_LOC "uart_rx" 18;
IO_PORT "uart_rx" IO_TYPE=LVCMOS33 PULL_MODE=UP;

// TMDS
IO_LOC "tmds_d_p[0]" 71,70;
IO_PORT "tmds_d_p[0]" PULL_MODE=NONE DRIVE=8;
IO_LOC "tmds_d_p[1]" 73,72;
IO_PORT "tmds_d_p[1]" PULL_MODE=NONE DRIVE=8;
IO_LOC "tmds_d_p[2]" 75,74;
IO_PORT "tmds_d_p[2]" PULL_MODE=NONE DRIVE=8;
IO_LOC "tmds_clk_p" 69,68;
IO_PORT "tmds_clk_p" PULL_MODE=NONE DRIVE=8;
\end{C++}
\end{table}

\subsection{表示のために SVO をブラックボックスとして使う}
HDMI へ表示するには SVO の svo\_hdmi\_top モジュールを使います(リスト\ref{svo-hdmi-top})。
AXI Stream のプロトコルに沿って tdata にアスキーコードを送れば
HDMI へ表示してくれるので、それをブラックボックスとして使います。

特徴的なのはクロックが3つあることでしょう。
clk は AXI Stream の処理用のクロック
clk\_pixel は描画用のクロックで、今回の出力解像度は VGA(640x480@60)
としたため 25.175Hz、
clk\_5x\_pixel は HDMI 用 で clk\_pixel の5倍です。

clk は画像処理をするためのクロックとも言え、ckl\_pixel より
速くすることで処理効率あげ、複雑な画像処理でも表示が要求するリアルタイム性に
追従するように設計することが可能です。

\begin{table}
\label{svo-hdmi-top}
\caption{svo\_hdmi\_top のインターフェース}

\begin{C++}
module svo_hdmi_top (
	input clk,
	input resetn,

	// video clocks
	input clk_pixel,
	input clk_5x_pixel,
	input locked,

	// AXI Stream 
	input term_in_tvalid,
	output term_out_tready,
	input [7:0] term_in_tdata,

	// output signals
	output       tmds_clk_n,
	output       tmds_clk_p,
	output [2:0] tmds_d_n,
	output [2:0] tmds_d_p
);
\end{C++}
\end{table}

\subsection{ピクセル・クロックを作るにはPLLを使う}
表示にはピクセル・クロックとして 25.175MHz と 125.875MHz が必要になります。
Tang Nano 9K では 27MHz のクロック入力があるので、
その1つの周波数から複数の周波数を生成します。
生成するにはFPGA内に予め備え付けられたPLL(Phase Locked Loop)\footnote{Gowin
の仕様書では rPLL}のプリミティブ(図\ref{PLL})\footnote{
PLL のプリミティブのI/Fは複雑です。Gowin IDE の
IPCore Generator を使うと必要なI/Fだけを持ったラッパー・モジュールを自動生成してくれます。
}
をIPコアから使います。

\figureimage{PLL}{PLL のプリミティブ}

\subsection{PLL のIPコアを設定する}
\footnote{詳しくは UG286-1.8J\_Gowin Clockユーザーガイド.pdf}
\figureimage{rPLL-ipcore}{PLL の IPコア}
IPCore Generator を使って PLL の IP コアを設定し、
必要なクロックを作るようにします
(図 \ref{rPLL-ipcore})。
CLKIN はシステムの 27MHz に Expected Frequency は 125.875 MHz に設定し、
Calculate ボタンを押します。すると、Actual Frequency の欄が 126MHz となり
エラーになってしまいます(図 \ref{pll-error})。
27MHz からは正確な 125.875MHz は作れないようです。そこで、生成される
クロック周波数の許容範囲を決める Tolerance として 0.2 \% を選択し
再度 Calculate とするとエラーがなくなります。

PLL の クロック出力は安定するまで時間がかかることがあります。
安定したことを示す lock 信号線も取り出せるように Eanble Lock の
チェックもしておきます。

画像処理を含めた全体の処理用に CLKOUTD から 63MHz を出力しています。
%CLKOUTD3 も有効にしていますが、今回は使っていません。

1つの PLL からはうまく 25.175MHz を作れないようなので
IP コアの CLKDIV も使いました(図 \ref{clkdiv})。
126MHz のクロックを 1/5 にしています。
これで 25.2MHz のクロックが出来たはずなので
VGA 用のピクセルクロックとします。
\figureimage{clkdiv}{CLKDIV IP コア}

クロックの設定が終わったらテスト的にトップモジュール内で配線をし、
HDMI へ正常に映像出力がなされるかどうかを確認します。
この時点ではまだ UART も Polyphony のモジュールも接続しておらず、
比較的複雑なクロック関連の確認をしていることになります(図\ref{hdmi-clock})。
うまく行けば、ディスプレーに SVO が標準で出力するテスト画像が表示されます。

\figureimage{hdmi-clock}{PLL を使った複数のクロック}

\subsection{Polyphony でのインタフェースとワーカー}
UART からアスキーコードを入力することはすでに出来ています。
また HDMI を通して VGA 出力もすでに出来ています。
この2つを合わせることで、Linux からキーボード入力すると
そのままディスプレに表示するように改造しましょう。

Polyphony で対応するインタフェースを作ります。class 名称を
\pythoninline{UART2AXIS} にしました(リスト \ref{polyphony-UART2AXIS})。

\begin{python}[label=polyphony-UART2AXIS, caption="Polyphony の UART2AXIS インタフェース"]
@module
class UART2AXIS:
    def __init__(self):
        self.i_tx_en = Port(bit, 'out', 0)
        self.waddr = Port(bit3, 'out', 0)
        self.wdata = Port(bit8, 'out', 0)

        self.i_rx_en = Port(bit, 'out', 0)
        self.raddr = Port(bit3, 'out', 0)
        self.rdata = Port(bit8, 'in')

        self.term_out_tvalid = Port(bool, 'out', 0)
        self.term_out_tdata = Port(bit8, 'out', 0)
        self.term_out_tready = Port(bool, 'in', 0)

        self.append_worker(self.main)
\end{python}

実際の処理は登録されたワーカー \pythoninline{main} で行います。
前章で使った \pythoninline{check_valid} と \pythoninline{read_char}
がそのまま使えます。
AXI Stream では tready が常に 1 であることを期待しており簡易的ですが、
この場合はうまく動作します。
\pythoninline{self.term_out_tvalid} を True にした後に 
\pythoninline{clkfence} せずに
(リストではコメントアウト)
False しています。
\pythoninline{self.term_out_tvalid} への連続する2度の書き込みは
順序よく処理しなければならず
同時には出来ないため
\pythoninline{clkfence} を使わなくても期待通りの動きとなります。
もちろん、書き方を統一する意図で
\pythoninline{clkfence} を書いても同じ動きとなります。

\begin{python}[label=UART2AXIS-main, caption="UART2AXIS の main ワーカー"]
    def main(self):
        while is_worker_running():
            if self.check_valid():
                data = self.read_char()

                self.term_out_tdata.wr(data)
                self.term_out_tvalid.wr(True)

                # clkfence()

                self.term_out_tvalid.wr(False)
\end{python}

